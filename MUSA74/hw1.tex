\documentclass[11pt]{article}
\usepackage{header}

\def\title{Homework 1 - Set Theory}

\begin{document}
\maketitle
\fontsize{12}{15}\selectfont

\begin{center}
  MUSA 74, Fall 2024 \\
  Due Sunday, September 8th at 11:59pm \\
\end{center}

\begin{quote}
  \textbf{Instructions:} This homework should to be submitted to
  Gradescope as a pdf file. Please work on separate sheets of paper and
  scan them, or type it up. We love it when students work together to
  complete assignments - if you choose to do this, please make sure to
  credit the person you worked with at the top of your page.
  
  \textbf{Problems:}
  
  1. Find sets $X$ and $Y$ such that $X$ is not a subset of
  $Y$ and $Y$ is not a subset of $X$.
  
  2. Suppose $X$ and $Y$ are finite sets where 
  $\lvert X\rvert=a$, 
  $\lvert Y\rvert=b$, and 
  $X\cap Y=\emptyset$.
  How many elements are in $X\cup Y$?
  
  % Suppose $X$ and $Y$ are finite sets such that
  % \emph{\textbar X\textbar{}} = \emph{a}, \emph{\textbar Y \textbar{}} =
  % \emph{b}, and \emph{X∩Y} = ∅(where ∅denotes the set with no elements).
  % How many elements are in \emph{X ∪Y} ? Explain your reasoning.
  
  % 3. Suppose \emph{A ⊆X}. Find simple descriptions of \emph{A ∪X} and
  % \emph{A ∩X} (you may use English or set builder notation). It may help
  % to draw out some Ven diagrams, or work through some examples (for
  % example, you could try computing \emph{A ∪X} and \emph{A ∩X} for
  % \emph{A} = \emph{\{}1\emph{,} 2\emph{\}} and \emph{X} =
  % \emph{\{}1\emph{,} 2\emph{,} 3\emph{,} 4\emph{\}}).
\end{quote}
  
  % 1
  
  \end{document}