%Setup%
\documentclass[12pt, letterpaper]{article}
\usepackage{math110-prep}
\title{LADR 2A}
\begin{document}
\maketitle

%Content%
\section*{$\R^n$ and $\C^n$}

\begin{imp}{2.3 Definition: linear combination}
A linear combination of a list of vectors in $V$ is a vector:
$$a_1v_1+\dots+a_mv_m,$$
where the scalars $\in\F$.
\end{imp}

\begin{imp}{2.5 Definition: span}
$$\Span(v_1,\dots,v_m)=\{\text{all possible linear combinations}\}$$
$$\Span()=\{0\}$$
\end{imp}

\begin{imp}{2.7 Span is the smallest containing subspace}
\begin{proof}
$\Span(v_1,\dots,v_m)\subset$ any subspace of V containing $v_1,\dots,v_m$ and span itself is a subspace of $V$. 
Therefore the smallest.
\end{proof}
\end{imp}

\begin{imp}{2.8 Definition: spans}
If $\Span(v_1,\dots,v_m)=V$, we say that $v_1,\dots,v_m$ spans $V$.
\end{imp}
\begin{imp}{2.10 Definition: finite-dimensional vector space}
A vector space is finite-dimensional if some list of vectors in it spans the space.
Every list has finite length (by definition).
\end{imp}

\begin{imp}{2.11 Definition polynomial $\poly(\F)$}
A function $p:\F\to\F$ is called a polynomial with coefficients $\F$ 
if there exists $a_0,\dots,a_m\in\F$ such that
$$p(z)=a_0+...+a_mz^m$$
for all $z\in\F$.
\\$\poly(\F)$ is the set of all polys w/ coefficients in $\F$.
\end{imp}

\begin{imp}{2.12 Definition degree of a polynomial}
Exists scalars $a_0,\dots,a_m\in\F$ w/ $a_m\neq0$ so:
$$p(z)=a_0+\dots+a_mz^m.$$
The poly = 0 has degree $-\infty$.
\end{imp}

\begin{imp}{2.13 Definition: $\poly_m(\F)$}
All polynomials w/ degree $\leq m$.
\end{imp}

\begin{imp}{infinite-dimensional vector space}
Not finite-dimensional.
\end{imp}

\begin{imp}{2.17 Definition: linear independent}
$v_1,\dots,v_m$ is linearly independent IFF each vector in
$\Span(v_1,\dots,v_m)$ has only 1 representation as a linear combination of
$v_1,\dots,v_m$. The most convenient condition that enables this is that there is only 1 
(the trivial solution) to make 0.
\end{imp}

\begin{imp}{2.19 Definition: linear dependent}
Not linearly independent.
\end{imp}

\begin{imp}{2.21 Definition: Linear Dependent Lemma}
Given a linear dependent list of vectors, you can remove a vector
without changing the span of the list.
\end{imp}

\begin{imp}{2.23 len(linearly independent list) $\leq$ len(spanning list)}
Trivial
\end{imp}

\begin{problem}{17}
Suppose $p_j(2)=0$ for each $j$. Prove $p_0\dots p_m$ is not linearly independent
in $\poly_m(F)$.
\end{problem}
\begin{proof}
Safely assume no 0 polynomials, or else it would be linearly dependent.
Factor out $(x-2)$ from all the polynomials into
$q_0\dots q_m$ in $\poly_{m-1}(F)$. Easy to see that this 
list of polynomials are not linearly independent in the new space.
\\(Better) Define polynomial $q\equiv1$. Then, $q\in\Span(p_0\dots p_m)$,
$$q(2)=a_0p_0(2)+\dots+a_mp_m(2).$$
Contradiction, 1 can't equal 0. Therefore $p_0...p_m$ can't be linearly independent.
\end{proof}

\end{document}