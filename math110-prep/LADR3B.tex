%Setup%
\documentclass[12pt, letterpaper]{article}
\usepackage{math110-prep}
\title{LADR 3B}
\begin{document}
\maketitle

%Content%
\section*{Null Spaces and Ranges}

\begin{imp}
{3.12 Definition: null space, $\Null T$}
For $T\in\Hom(V,W),\Null T=\{v\in V:Tv=0\}$
\end{imp}

\begin{imp}
{3.15 Definition: injective (aka one to one)}
A function $T:V\to W$ is injective if 
$$Tu=Tv\implies u=v$$
It may be easier to think of the contrapositive. 
In english: $T$ is injective
if it maps distinct inputs to distinct outputs.
\end{imp}

\begin{imp}
{3.16 Injectivity is equivalent to null space = \{0\}}
($\rightarrow$)
Since injective, $T(x)=0$ is only true for one $x$ value.
$T(x)=T(0)+T(x)\implies T(0)=0$. That $x$ must be 0.
Therefore, injective$\implies \Null T=\{0\}$.
\\($\leftarrow$)
$T(u)=T(v)\implies T(u-v)=0$.
Since $\Null T=\{0\}$, $u-v=0\implies u=v$.
\end{imp}

\begin{imp}
{3.17 Definition: range}
For $T:V\to W$, $\Range T$ is subset of $W$
w/ form $Tv$.
$$\Range T=\{Tv:v\in V\}$$
\end{imp}

\begin{imp}
{3.19 Range is a subspace}
$$T(0)=0\implies 0\in \Range T$$
$$T(v_1+v_2) = Tv_1+Tv_2 = w_1+w_2$$
$$T(\lambda v)=\lambda Tv=\lambda w$$
\end{imp}

\begin{imp}
{3.20 Definition: surjective (aka onto)}
A function $T:V\to W$ is called surjective 
if its range equals $W$.
\end{imp}

\begin{imp}
{3.22 Fundamental Theorem of Linear Maps}
$$\dim V=\dim\Null T+\dim\Range T$$
\begin{proof}
Let $u_1,\dots,u_m$ is a basis of $\Null T$.
Extend this to a basis of $V$: 
$$u_1,\dots,u_m,v_1,\dots,v_n.$$
Then, $\dim V=m+n$. Now, we need to show $\Range T=n$ and finite dimensional.
Let $$v\in V, v=a_1u_1+\dots+a_mu_m+b_1v_1+\dots+b_nv_n.$$ 
$$Tv=b_1Tv_1+\dots+b_nTv_n.$$
Indeed, $\Range T=n$ and finite dimensional.
\end{proof}
\end{imp}

\begin{imp}
{3.23 A map to a smaller dimensional space is not injective}
$\dim V>\dim W\implies T:V\to W$ is not injective.
\begin{proof}
$$\dim V>\dim W\implies \dim V-\dim W\geq 1$$
$$\dim V-\dim\Range T\geq1\implies \dim\Null T\geq 1$$
Then, $\Null T\neq\{0\}$, and thus $T$ is not injective.
\end{proof}
\end{imp}

\begin{imp}
{3.24 A map to a larger dimensional space is not surjective}
$\dim V<\dim W\implies T:V\to W$ is not surjective.
\begin{proof}
$$\dim V-\dim W<0$$
$$\dim V=\dim\Null T+\dim\Range T\implies\dim V-\dim\Range T\geq0$$
$$\dim\Range T\leq\dim V<\dim W\implies\dim\Range T<\dim W$$
Since the dimensions of $\Range T$ and $W$ are different, they are not the same vector space.
Therefore, $T$ is not surjective.
\end{proof}
\end{imp}

\begin{imp}
{3.25 Linear system of equations}
$n$ variables, $m$ equations:
$$\sum_{k=1}^nA_{1,k}x_k = c_1$$
$$\vdots$$
$$\sum_{k=1}^nA_{m,k}x_k = c_m$$
$T(x_1,\dots,x_n)=(
\sum_{k=1}^nA_{1,k}x_k,
\dots,
\sum_{k=1}^nA_{m,k}x_k)=(c_1,\dots,c_m)\implies
T:\F^n\to\F^m$

\end{imp}

\begin{imp}
{3.26 Definition: Homogeneous system of linear equations}
System of linear equations where constant on RHS
of each equation is zero.
\\Property (similar to 3.23): If more variables than
equations, their are multiple solutions that lead to 0 (not injective).
\end{imp}

\begin{imp}
{3.29 Definition: Inhomogeneous}
 Not homogeneous, vector of constants is non-zero.
\\Property (similar to 3.24): If less variables than
equations, there are constants with no corresponding solutions (not surjective).
\end{imp}

\begin{problem}
{1} Give an example of a linear map $T$ such that 
$\dim\Null T=3$ and $\dim\Range T=2$.
\end{problem}
\begin{proof}
$$T(e_1,e_2,e_3,e_4,e_5)=(e_1,e_2,0,0,0)$$
\end{proof}

\begin{problem}
{2} Suppose $V$ is a vector space and $S,T\in\Hom(V,V)$ are such that
$$\Range S\subset\Null T.$$
Prove that $(ST)^2=0$.
\end{problem}
\begin{proof}
Since $\Range S$ is a subset of $\Null T$, 
$T(S(v))=0$ for all $v\in V$. Then the composed 
linear map $TS=0$. Then,
$$(ST)^2=S(TS)T=S(0)T=0$$.
\end{proof}

\begin{problem}
{5} Give an example of a linear map $T:\R^4\to\R^4$ such that
$$\Range T=\Null T.$$
\end{problem}
\begin{proof}
Notice that the given condition implies: $$\forall v\in\R^4,T(T(v))=0$$
and that nothing outside of $x=T(v)$ for $T(x)$ yields 0.
What about a linear map that shifts the vector to the left by 2?
$$T(x_1,x_2,x_3,x_4)=x_3,x_4,0,0$$
$$\Null T=\Range T=\R^2\times\{0\}^2$$
\end{proof}

\begin{problem}
{6} Prove that there does not exist a linear map $T:\R^5\to\R^5$ such that
$$\Range T=\Null T.$$
\end{problem}
\begin{proof}
By Fundamental Theorem of Linear Maps,
$$\dim T=\dim\Range T+\dim\Null T.$$
Since both the range and null space are the same it follows that
$$\dim\Range T= \dim\Null T.$$
Then, $\dim\Range T= \frac{\dim T}{2} = 2.5$, which is impossible.
\end{proof}

\begin{problem}
{7}Show that 
$\{T\in\Hom(V,W):T \text{ is not injective}\}$ is not a subspace.
\end{problem}
\begin{proof}
Let $S,T$ be not injective linear maps.
$$S(v_1)\neq0,S(v_2)=0,T(v_1)=0,T(v_2)\neq0$$
$$(S+T)(v_1)=S(v_1)+T(v_1)=S(v_1)\neq0$$
\end{proof}

\begin{problem}
{8}Show that 
$\{T\in\Hom(V,W):T \text{ is not surjective}\}$ is not a subspace.
\end{problem}
\begin{proof}
$$S(v_1,v_2)=(v_1,0),T(v_1,v_2)=(0,v_2)\implies(S+T)(v_1,v_2)=(v_1,v_2)$$
\end{proof}

\begin{problem}
{31} Give an example of two linear maps 
$T_1,T_2:\R^5\to\R^2$ that have the same null space but are not scalar multiples of each other.
\end{problem}
\begin{proof}
$$T_1(v_1,v_2,v_3,v_4,v_5)=(v_1,v_2,0,0,0)$$
$$T_2(v_1,v_2,v_3,v_4,v_5)=(v_2,v_1,0,0,0)$$
$$e_1\neq\lambda e_2$$
\end{proof}



\end{document}