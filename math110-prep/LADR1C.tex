%Setup%
\documentclass[12pt, letterpaper]{article}
\usepackage{math110-prep}
\title{LADR 1C}
\begin{document}
\maketitle

%Content%
\section*{Subspaces}

\begin{imp}{1.32 Definition subspace}
A subset $U$ of $V$ is called a subspace of $V$ if $U$ is a vector space.
\end{imp}

\begin{imp}{1.34 Conditions of subspace}
\begin{itemize}
    \item additive identity: 
    $0\in U$
    \item closed under addition: 
    $u,w\in U \implies u+w\in U$
    \item closed under scalar multiplication:
    $a\in F\land u\in U \implies au\in U$
\end{itemize}
\end{imp}

\begin{imp}{1.35 Examples}
Solved them on paper separately.
\end{imp}

\begin{imp}{1.36 Definition sum of subsets}
$$U_1+U_2=\{u_1+u_2:u_1\in U_1,u_2\in U_2\}$$
\end{imp}

\begin{imp}{1.40 Definition direct sum}
The sum $U_1 + ... + U_m$ is a direct sum if each element in 
the sum can be written in only one way as a sum 
$u_1 + ... u_m$ where $u_j \in U_j$.
\end{imp}

\begin{imp}{1.44 Condition for a direct sum}
IFF there is only 1 way to create 0: each $u_j = 0$.
\begin{proof} (Me)
\\Let there be multiple ways to make 0.
$$x=U_1(x) + ... + U_m(x)=U_1(x+0) + ... + U_m(x+0)$$
Then there are multiple ways to make $x$.
\\Conversely, let there be multiple ways to make x.
$$U_1(x)+...+U_m(x)=U_1(y)+...+U_m(y)$$
$$U_1(x-y)+...+U_m(x-y)=0$$
Therefore, multiple ways to make x IFF multiple ways to make 0.
Finally, 1 way to make $x$ IFF 1 way to make 0.
\end{proof}
\begin{proof} (Book)
$$v = u_1 + ... + u_m = v_1 + ... + v_m$$
$$0 = (u_1 - v_1) + ... + (u_m - v_m)$$
If there is only 1 way to make 0, $u_j = v_j$ is a solution and the only solution.
\end{proof}
\end{imp}

\begin{imp}{1.45 Direct sum of 2 subspaces}
IFF $U \cap W = \{0\}$
\begin{proof} (Simple)
If more elements in the intersection, then multiple ways to make 0.
If multiple ways to make 0, then need at least one other element in the intersection.
\end{proof}
\end{imp}

\begin{problem}{1}
(a) $\{(x_1,x_2,x_3)\in\F^3:x_1+2x_2+3x_3=0\}$
\begin{proof}
$(0,0,0)$ exists.
$(x_1,x_2,x_3) + (y_1,y_2,y_3) \implies (x_1+y_1)+2(x_2+y_2)+3(x_3+y_3)$ exists.
$c(x_1,x_2,x_3) \implies (cx_1)+2(cx_2)+3(cx_3)$ exists.
\end{proof}
(b) $\{(x_1,x_2,x_3)\in\F^3:x_1+2x_2+3x_3=4\}$
\begin{proof}
We can easily see this is not closed under addition.
\end{proof}
(c) similar to (b)\\
(d) similar to (a)
\end{problem}

\begin{problem}{3}
Show that the set of differential real-valued functions on $(-4,4)$
where $f'(-1)=3f(2)$ is a subspace of $\R^{(-4,4)}$.
\end{problem}
\begin{proof}
$$f(x) = 0 \implies f'(-1) = 0 = 3f(2)$$
$$(f+g)'(-1)=f'(-1)+g'(-1)=3f(2)+3g(2)=3(f+g)(2)$$
$$c(f'(-1))=c(3f(2))\implies (cf')(-1)=3(cf)(2)$$
\end{proof}

\begin{problem}{5}
Is $\R^2$ a subspace of $\C^2$?
\end{problem}
\begin{proof}
Not closed under multiplication: $\forall x\in \R: ix\notin\R$
\end{proof}

\end{document}