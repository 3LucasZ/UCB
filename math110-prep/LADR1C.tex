%Setup%
\documentclass[12pt, letterpaper]{article}
\usepackage{math110-prep}
\title{LADR 1C}
\begin{document}
\maketitle

%Content%
\section*{Subspaces}

\begin{imp}{1.32 Definition subspace}
A subset $U$ of $V$ is called a subspace of $V$ if $U$ is a vector space.
\end{imp}

\begin{imp}{1.34 Conditions of subspace}
\begin{itemize}
    \item additive identity: 
    $0\in U$ (Note: this is simply the easiest way to check $U$ has at least 1 element in it.
    Closure under scalar multiplication already implies a 0 element exists given a non-empty space.)
    \item closed under addition: 
    $u,w\in U \implies u+w\in U$
    \item closed under scalar multiplication:
    $a\in F\land u\in U \implies au\in U$
\end{itemize}
\end{imp}

\begin{imp}{1.35 Examples}
Solved them on paper separately.
\end{imp}

\begin{imp}{1.36 Definition sum of subsets}
$$U_1+U_2=\{u_1+u_2:u_1\in U_1,u_2\in U_2\}$$
\end{imp}

\begin{imp}{1.40 Definition direct sum}
The sum $U_1 + ... + U_m$ is a direct sum if each element in 
the sum can be written in only one way as a sum 
$u_1 + ... u_m$ where $u_j \in U_j$.
\end{imp}

\begin{imp}{1.44 Condition for a direct sum}
IFF there is only 1 way to create 0: each $u_j = 0$.
\begin{proof} (Me)
\\Let there be multiple ways to make 0.
$$x=U_1(x) + ... + U_m(x)=U_1(x+0) + ... + U_m(x+0)$$
Then there are multiple ways to make $x$.
\\Conversely, let there be multiple ways to make x.
$$U_1(x)+...+U_m(x)=U_1(y)+...+U_m(y)$$
$$U_1(x-y)+...+U_m(x-y)=0$$
Therefore, multiple ways to make x IFF multiple ways to make 0.
Finally, 1 way to make $x$ IFF 1 way to make 0.
\end{proof}
\begin{proof} (Book)
$$v = u_1 + ... + u_m = v_1 + ... + v_m$$
$$0 = (u_1 - v_1) + ... + (u_m - v_m)$$
If there is only 1 way to make 0, $u_j = v_j$ is a solution and the only solution.
\end{proof}
\end{imp}

\begin{imp}{1.45 Direct sum of 2 subspaces}
IFF $U \cap W = \{0\}$
\begin{proof} (Simple)
If more elements in the intersection, then multiple ways to make 0.
If multiple ways to make 0, then need at least one other element in the intersection.
\end{proof}
\end{imp}

\begin{problem}{1}
(a) $\{(x_1,x_2,x_3)\in\F^3:x_1+2x_2+3x_3=0\}$
(b) $\{(x_1,x_2,x_3)\in\F^3:x_1+2x_2+3x_3=4\}$
(c) similar to (b)\\
(d) similar to (a)
\end{problem}
\begin{proof} (a)
$(0,0,0)$ exists.
$(x_1,x_2,x_3) + (y_1,y_2,y_3) \implies (x_1+y_1)+2(x_2+y_2)+3(x_3+y_3)$ exists.
$c(x_1,x_2,x_3) \implies (cx_1)+2(cx_2)+3(cx_3)$ exists.
\end{proof}
\begin{proof} (b)
We can easily see this is not closed under addition.
\end{proof}

\begin{problem}{3}
Show that the set of differential real-valued functions on $(-4,4)$
where $f'(-1)=3f(2)$ is a subspace of $\R^{(-4,4)}$.
\end{problem}
\begin{proof}
$$f(x) = 0 \implies f'(-1) = 0 = 3f(2)$$
$$(f+g)'(-1)=f'(-1)+g'(-1)=3f(2)+3g(2)=3(f+g)(2)$$
$$c(f'(-1))=c(3f(2))\implies (cf')(-1)=3(cf)(2)$$
\end{proof}

\begin{problem}{5}
Is $\R^2$ a subspace of $\C^2$?
\end{problem}
\begin{proof}
Not closed under multiplication: $\forall x\in \R: ix\notin\R$
\end{proof}

\begin{problem}{7}
Give an example of a subset $U$ of $\R^2$ 
where $U$ is closed under addition and additive inverses.
\end{problem}
\begin{proof}
Since it's closed under addition, it's closed under integer scalar multiplication.
However, we need it to be not closed under float scalar multiplication to not be a subspace.
The subset $U=\Z^2$ fits the conditions nicely. 
\end{proof}

\begin{problem}{8}
Closed under scalar multiplication, but not a subset?
\end{problem}
\begin{proof}
$$\{(x,0):x\in\R\}\cup\{(0,y):y\in\R\}$$
\end{proof}

\begin{problem}{9}
Set of periodic functions $\R\to\R$ subspace of $\R^\R$?
\end{problem}
\begin{proof}
Let $f(x)$ be a function with period 2: $\sin(2\pi/2)$ and $g(x)$ be a function with period $\sqrt{2}$: $\sin(2\pi/\sqrt{2}$).
In order for $(f+g)(x)$ to be periodic as well, there must be infinite $x$ where
$f(x)$ and $g(x)$ are at their maximas. There must be a pair of integers $a,b$ where
$a\cdot2=b\cdot\sqrt{2}$. This is impossible, and thus by contradiction, the set is not closed under addition and thus not a subspace.
\end{proof}

\begin{problem}{10}
Prove $U_1\cap U_2$ is a subspace of $V$.
\end{problem}
\begin{proof}
First, we prove non-empty intersection.
$$0\in U_1\land0\in U_2\implies0\in U_1\cap U_2$$
Next, we prove closed under addition. Assume $x,y\in U_1\cap U_2\land x+y\notin U_1\cap U_2$. 
However, by definition of subspace, $x+y\in U_1\land x+y\in U_2$. 
Thus contradiction; truly, $x+y\in U_1\cap U_2$.
Similarly, we can prove closed under multiplication.
\end{proof}

\begin{problem}{11}
Prove that the intersection of every collection of subspaces of $V$ is a subspace of $V$.
\end{problem}
\begin{proof}
Let $\mathfrak{C}$ denote some collection of subspaces of $V$. Let
$$U = \bigcap_{W\in \mathfrak{C}}W.$$
Solution should be very similar to problem 10.
\end{proof}

\begin{problem}{12}
Prove $U_1\cup U_2$ is a subspace IFF one is contained in the other.
\end{problem}
\begin{proof}
WLOG, if one subspace contains the other, let the superset be $U_1$ and $U_1\cup U_2=U_1$.
\\Forward direction is trivial: $U_1$ is a subspace.
\\Conversely, we prove $U_1\cup U_2$ is closed under addition IFF $U_1\supset U_2$. Let
$$x\in U_1\land x\notin U_2 \land y\in U_2\land y\notin U_1$$
$$\implies x+y\notin U_1 \land x+y\notin U_2$$
$$\implies x+y\notin U_1\cap U_2.$$
By contraposition: $x+y\in U_1\cup U_2\implies U_1\supset U_2$. Thus, we have proved IFF.
\end{proof}

\begin{problem}{13}
Prove $U_1\cup U_2\cup U_3$ is a subspace IFF one contains the other 2.
\end{problem}
\begin{proof}
Similar to (12)
\end{proof}

\begin{problem}{15}
What is $U+U$?
\end{problem}
\begin{proof}
By definition,
$$U+U=\{u_1+u_2:u_1\in U,u_2\in U\}$$
Since $U$ is closed under addition,
$$\implies u_1+u_2\in U \implies U+U=U$$
\end{proof}

\begin{problem}{16}
Is addition on subspaces of V commutative?
\end{problem}
\begin{proof}
Trivial: yes
\end{proof}

\begin{problem}{17}
Associative?
\end{problem}
\begin{proof}
Trivial: yes
\end{proof}

\begin{problem}{25}
Let $U_e$ denote the set of real-valued even functions
and let $U_o$ denote the set of real-valued odd functions.
Prove $U_e\oplus U_o=\R^\R$.
\end{problem}
\begin{proof}
Let 
$$h\in\R^\R,f(x)=\frac{h(x)+h(-x)}{2},g(x)=\frac{h(x)-h(-x)}{2}.$$
$f(-x)=f(x)$, and is thus an even function.
$g(-x)=-g(x)$, and is thus an odd function.
$f(x)+g(x)=h(x)$, and thus every function $h\in\R^\R$ can be represented as the sum of a even and odd function.
\end{proof}
Below are some points that led me to the solution.
Even and odd functions are free functions for $x\geq0$ but have a fixed behavior for $x<0$.
Is it possible to prove that all functions $\R^\R$ can be represented as the sum of an even and odd function?
This would be the same thing as saying, for $x\geq0$, $h(x)=f(x)+g(x)$ and $h(-x)=f(x)-g(x)$.

\end{document}