%Setup%
\documentclass[12pt, letterpaper]{article}
\usepackage{math110-prep}
\title{LADR 1B}
\begin{document}
\maketitle

%Content%
\section*{Definition of Vector Space}
\begin{imp}{1.18 Definition of addition, scalar multiplication on V}
\ldots
\end{imp}

\begin{imp}{1.19 Definition of a vector space}
A vector space is a set V along with an addition on V and a scalar multiplication on V such that:
commutativity, associativity, additive inverse, multiplicative identity, distributive properties
all hold.
\end{imp}
    
\begin{imp}{1.20 Definition of vector, point}
Vectors or points refer to elements of a vector space.
\end{imp}
    
Scalar multiplication in $V$ depends on $\mathbb{F}$. When we need to be precise, we
say that $V$ is a vector space over $\mathbb{F}$. Usually it is obvious from context
or irrelevant though.
    
\begin{imp}{1.21 Definition of real, complex vector spaces}
\ldots
\end{imp}
    
\begin{imp}{1.23 Notation $\mathbb{F}^S$}
\begin{itemize}
    \item $\mathbb{F}^S$ is the set of functions w/ domain set $S$ and range set $\mathbb{F}$.
    \item (Fake explanation) Such a function has $|S|$ possible inputs and $|F|$ possible outputs.
    There are then $|F|^{|S|}$ functions that obey this rule which motivates the notation.
    \item \href{https://math.stackexchange.com/questions/4358397/why-fs-represents-set-of-functions-from-s-to-f}{More info here}
    \item For $f,g,f+g\in\mathbb{F}^S$,
    $$(f+g)(x) = f(x) + g(x)$$
    \item For $\lambda\in\mathbb{F}$ and $f,\lambda f\in\mathbb{F}^S$,
    $$(\lambda f)(x)=\lambda f(x)$$
\end{itemize}
\end{imp}

\begin{problem}{1}						% the Proposition number from the book (this one is fictitious)
Prove that $-(-v)=v$ for every $v\in V$																				% the Proof starts with a new paragraph
\end{problem}

\begin{proof}       						% makes the word Proof bold face
We want to show that $v$ is the additive inverse of $(-v)$. We have 
$$(-v)+v=(-1)v+1v=(-1+1)v=0v=0,$$
as desired.
\end{proof}

\begin{problem}{3}
Suppose $v,w\in V$. Explain why there exists a unique $x\in V$ such that $v+3x=w$.
\end{problem}

\begin{proof}
Let $x=\frac{1}{3}(w-v)$. This
$$v+3x=v+(w-v)=w,$$
proving existence.
Let $y\in V$ where $v+3y=w$. Then
$$v+3y=v+3x\iff 3y=3x\iff y=x,$$
proving uniqueness.
\end{proof}

\begin{problem}{5}
Show that the additive inverse condition on vector spaces (1.19) 
can be replaced with the condition that
$$0v=0 \text{ for all } v\in V.$$
\end{problem}

\begin{proof}
We show that the 2 statements are equivalent.\\
First, we prove the old condition implies the new condition.
Assume that every $v\in V$ has an additive inverse.
We have
$$0v+0v=(0+0)v=0v.$$
Adding the additive inverse of 0v on both sides yields
$0v=0$ as desired.\\
Second, we prove the new condition implies the old condition.
Assume that $0v=0$ for all $v\in V$.
We have
$$v+(-1)v=(1+(-1))v=0v=0.$$
Hence, every element has an additive inverse, as desired.
\end{proof}

\begin{problem}{6}
Is $\R\cup\{\infty\}\cup\{-\infty\}$ a vector space over $\R$?
\end{problem}

\begin{proof}
For a set to be a vector space, it must follow 6 conditions:
commutativity, associativity, additive identity, additive inverse, multiplicative identity, and distributive properties.
We will try to find a counter-example to break one of these rules.
$$(\infty+(-\infty))+3=0+3=3$$
$$\infty+((-\infty)+3)=\infty+(-\infty)=0$$
Since these 2 expressions are not equal to each other, we have shown that the rule of associativity
is broken in the set given. Therefore, it is not a vector space over $\R$.
\end{proof}

\end{document}