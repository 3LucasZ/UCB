%Setup%
\documentclass[12pt, letterpaper]{article}
\usepackage{math110-prep}
\title{LADR 3C}
\begin{document}
\maketitle

%Content%
\section*{Matrices}

\begin{imp}
{3.30 Definition: matrix, $A_{j,k}$}
An $m$-by-$n$ matrix $A$ is a rectangular array of elements of $F$
$$A=\begin{pmatrix}
A_{1,1} & \dots & A_{1,n}\\
\vdots & { } & \vdots\\
A_{m,1} & \dots & A_{m,n}
\end{pmatrix}$$
$A_{j,k}=A[j][k]$ like a software 2D array.
\end{imp}

\begin{imp}
{3.32 Definition: matrix of a linear map, $\mat(T)$}
Suppose $T\in\Hom(V,W)$ and 
$v_1,\dots,v_n$ is a basis of $V$ and 
$w_1,\dots,w_m$ is a basis of $W$.
$A=\mat(T)$ where $Tv_k=A_{1,k}w_1+\dots+A_{m,k}w_m$. 
$$\mat(T)=
\begin{matrix}
w_1\\\vdots\\w_m
\end{matrix}
\stackrel{
\begin{matrix}
  v_1 & \dots & v_k & \dots & v_m
\end{matrix}
}{
\begin{pmatrix}
{A_{1,1}} & {\dots} & A_{1,k} & {\dots} & {A_{1,n}}\\
{\vdots} & { } & \vdots & { } & {\vdots}\\
{A_{m,1}} & {\dots} & A_{m,k} & {\dots} & {A_{m,n}}
\end{pmatrix}
}$$
\end{imp}
Unless explicitly stated, assume the bases used are the standard ones.
You can imagine elements of $\F^m$ as columns of $m$ numbers.
Then, $k^{th}$ column of $\mat(T)$ is $T$ applied to the $k^{th}$ standard basis vector.

\begin{imp}
{3.35 Definition: matrix addition}
Matrix of sum of linear maps. A, C must be same size. 
$(A+C)_{j,k}=A_{j,k}+C_{j,k}$
\end{imp}

\begin{imp}
{3.37 Definition: scalar multiplication of a matrix}
Matrix of a scalar times a linear map. 
$(\lambda A)_{j,k}=\lambda A_{j,k}$
\end{imp}

\begin{imp}
{3.39 Notation: $\F^{m,n}$}
Set of all m by n matrices with entries in $\F$.
\end{imp}

\begin{imp}
{3.40 $\F^{m,n}$ is a vector space}
Closed under addition and multiplication (3.35, 3.37). 
It is clearly a non-empty set, the 0 element is the matrix with all 0 entries.
\\Property: $\dim\F^{m,n}=mn$
\end{imp}

\begin{imp}
{3.41 Definition: matrix multiplication}
We want an operator so that the equality holds: $$\mat(ST)=\mat(S)\mat(T).$$
Let $A$ be m by n, $C$ be n by p. Then, $AC$ is m by p.
$$(AC)_{j,k}=\sum_{r=1}^nA_{j,r}C_{r,k}$$
Another way:
$$(AC)_{j,k}=A_{j,\cdot}C_{\cdot,k}$$
\end{imp}

\begin{imp}
{3.44 Notation: $A_{j,\cdot},A_{\cdot,k}$}
$A_{j,\cdot}$ is 1 by n matrix for row j of $A$.\\
$A_{\cdot,k}$ is m by 1 matrix for col k of $A$.
\end{imp}

\begin{imp}
{3.49 Column of matrix product equals matrix times column}
$$(AC)_{\cdot,k}=AC_{\cdot,k}$$
\end{imp}

\begin{imp}
{3.52 Linear combination of columns}
$$Ac=c_1A_{\cdot,1}+\dots+c_nA_{\cdot,n}$$
\end{imp}

\end{document}