%Dependencies%
\documentclass[12pt, letterpaper]{article}
\usepackage{amsmath}
\usepackage{amssymb}
\usepackage{fancyhdr}
\DeclareMathOperator\cis{cis}

%Setup$
\title{LADR 1A Exercises}
\author{Lucas Zheng}
\date{}
\pagestyle{fancy}
\fancypagestyle{plain}{}
\fancyhf{} 
\rfoot{Page \thepage}
\begin{document}
\maketitle

%Content%
\textbf{1Q.} 
Suppose $a$ and $b$ are real numbers, not both 0. Find real numbers $c$ and
$d$ such that
$${\frac{1}{a+bi}}=c+di.$$

\textbf{1A.} 
Multiply the numerator and denominator of the fraction by the conjugate of the denominator
$$\frac{1}{a+bi} = \frac{a-bi}{a^2+b^2}.$$
Expand the fraction
$$\frac{a-bi}{a^2+b^2} = \frac{a}{a^2+b^2} - \frac{b}{a^2+b^2}i.$$
We see a form arise similar to that of ${c+di}$. Let
$$c = \frac{a-bi}{a^2+b^2}$$
$$d = -\frac{b}{a^2+b^2}$$
and we're done.

\textbf{2Q.} 
Show that
$$\frac{-1+\sqrt{3}i}{2}$$
is a cube root of 1 (meaning that its cube equals 1).

\textbf{2A.} 
Use euler form to represent the expression as
$$\cis(\frac{2}{3}\pi).$$
Apply DeMoivre's theorem to cube the expression
$$\cis(\frac{2}{3}\pi)^3 = \cis(3\cdot\frac{2}{3}\pi) = \cos(\pi) + \sin(\pi)i = 1.$$
Hence, it is indeed a cube root of 1.

\textbf{3Q.} 
Find two distinct square roots of $i$.

\textbf{3A.} 
Suppose there exists $a,b\in\R$ where $(a+bi)^2 = i$. Expand the expression into
$$(a^2-b^2)+(2ab)i = i.$$
We split this into 2 equalities
$$a^2-b^2=0$$
$$2ab = 1.$$
There are 2 cases that satisfy the first equation: $a = b$ and $a = -b$.
\\Case 1: $a=b$
\\Then
$$2a^2 = 1$$
$$a = \pm\sqrt{\frac{1}{2}}.$$
We yield 2 solutions in this case.
\\Case 2: $a=-b$
\\Then
$$-2a^2 = 1$$
$$a = \pm\sqrt{-\frac{1}{2}}.$$
We yield 0 solutions in this case, since the condition we set in the beginning is that $a\in\R$.
Hence, there are 2 and only 2 distinct roots of i
$$\pm\sqrt{\frac{1}{2}}(1+i).$$

\textbf{4Q.}
Show that $\alpha + \beta = \beta + \alpha$ for all $\alpha,\beta\in\C$.

\textbf{4A.}
Let $\alpha = a + bi$ and $\beta = c + di$ for $a,b,c,d\in\R$. Then
$$\alpha + \beta = a + bi + c + di = c + di + a + bi = \beta + \alpha.$$

\textbf{5Q.} Show that $(\alpha+\beta)+\gamma=\alpha+(\beta+\gamma)$ for all $\alpha,\beta,\gamma\in\C$.

\textbf{5A.} Trivial.

\textbf{6Q.} Show that $(\alpha\beta)\gamma=\alpha(\beta\gamma)$ for all $\alpha,\beta,\gamma\in\C$.

\textbf{6A.} Trivial.

\textbf{7Q.} Show that for every $\alpha\in\C$, there exists a unique $\beta\in\C$ such that $\alpha+\beta$ = 0.

\textbf{7A.} Let $\alpha=a_1+a_2i$ for some $a_1,a_2\in\R$ and let $\beta = -a_1-a_2i$. Then
$$\alpha+\beta=(a_1-a_1)+(b_1-b_1)i=0,$$
proving existence. 
\\Let $\gamma\in\C$ where $\alpha+\gamma=0$. Then
$$\gamma = \gamma + (\alpha + \beta) = (\gamma + \alpha) +\beta = 0 + \beta = \beta,$$
proving uniqueness.

\end{document}