\documentclass[12 pt]{article} %sets the font to 12 pt and says this is an article (as opposed to book or other documents)
\usepackage{amssymb, amsmath}
\pagestyle{myheadings} % tells LaTeX to allow you to enter information in the heading
\markright{Lucas Zheng \hfill \today \hfill} 
\newcommand{\eqn}[0]{\begin{array}{rcl}}%begin an aligned equation - allows for aligning = or inequalities.  Always use with $$ $$
\newcommand{\eqnend}[0]{\end{array} }  	%end the aligned equation
\newcommand{\qed}[0]{$\square$}        	% make an unfilled square the default for ending a proof

\begin{document}												% end of preamble and beginning of text that will be printed
        																% makes the word Proposition and the proposition number bold face  
\textbf{Q1.}							% the Proposition number from the book (this one is fictitious)
Prove that $-(-v)=v$ for every $v\in V$																				% the Proof starts with a new paragraph

\textbf{Proof.}              						% makes the word Proof bold face
We want to show that $v$ is the additive inverse of $(-v)$. We have 
$$(-v)+v=(-1)v+1v=(-1+1)v=0v=0,$$
as desired. $\square$

\textbf{Q3.}
Suppose $v,w\in V$. Explain why there exists a unique $x\in V$ such that $v+3x=w$.

\textbf{Proof.}
Let $x=\frac{1}{3}(w-v)$. This
$$v+3x=v+(w-v)=w,$$
proving existence.
Let $y\in V$ where $v+3y=w$. Then
$$v+3y=v+3x\iff 3y=3x\iff y=x,$$
proving uniqueness. $\square$

\textbf{Q5.}
Show that the additive inverse condition on vector spaces (1.19) 
can be replaced with the condition that
$$0v=0 \text{ for all } v\in V.$$

\textbf{Proof.}
We show that the 2 statements are equivalent.\\
First, we prove the old condition implies the new condition.
Assume that every $v\in V$ has an additive inverse.
We have
$$0v+0v=(0+0)v=0v.$$
Adding the additive inverse of 0v on both sides yields
$0v=0$ as desired.\\
Second, we prove the new condition implies the old condition.
Assume that $0v=0$ for all $v\in V$.
We have
$$v+(-1)v=(1+(-1))v=0v=0.$$
Hence, every element has an additive inverse, as desired. $\square$

\textbf{Q6.}
Is $\mathbb{R}\cup\{\infty\}\cup\{-\infty\}$ a vector space over $\mathbb{R}$?

\textbf{Proof.}
For a set to be a vector space, it must follow 6 conditions:
commutativity, associativity, additive identity, additive inverse, multiplicative identity, and distributive properties.
We will try to find a counter-example to break one of these rules.
$$(\infty+(-\infty))+3=0+3=3$$
$$\infty+((-\infty)+3)=\infty+(-\infty)=0$$
Since these 2 expressions are not equal to each other, we have shown that the rule of associativity
is broken in the set given. Therefore, it is not a vector space over $\mathbb{R}$. $\square$

\end{document}