%Setup%
\documentclass[12pt, letterpaper]{article}
\usepackage{cs70}
\title{CS70 Note 2: Proofs}
\begin{document}
\maketitle

%Content%
\section{Proof Basics}
A proof is a finite sequence of steps called logical deductions which establishes the truth of a statement.
There are certain statements called axioms/postulates that we accept without proof.
We apply the rules of logic to formulate complex truths.
\\\textbf{Direct proof} 
assume $P$ \dots therefore $Q$.
\\Ex: $x$ divisible by 9 IFF the sum of its digits are also divisible by 9.
\\\textbf{Proof by contraposition}
assume $\lnot Q$ \dots therefore $\lnot P$
\\Ex: Pigeonhole principle
\\\textbf{Proof by contradiction}
assume $\lnot P$ \dots $R$ \dots $\lnot R$ therefore $P$
\\Ex: "There are infinitely many primes"
\\\textbf{Proof by cases} we don't know which of a set of possible cases
is true, but we know at least one case is true. Prove the result in both cases
then the general statement holds.
\\Ex: there exists irrational $x,y$ so $x^y$ is rational.
\begin{proof}
Demonstrate a single $x,y$ pair works is sufficient. 
Let $x=\sqrt2$ and $y=\sqrt2$. Divide the proof into 2 cases, one case must be true.
\begin{itemize}
  \item $\sqrt2^{\sqrt2}$ is rational
  \\Immediately yields our claim, rational.
  \item $\sqrt2^{\sqrt2}$ is irrational
  \\Try a different irrational x, $x=\sqrt2^{\sqrt2}$.
  $$(\sqrt2^{\sqrt2})^{\sqrt2}=\sqrt2^2=2$$
  which is rational.
\end{itemize}
\end{proof}
To me, this is oddly beautiful. We don't know what the actual $(x,y)$ pair
satisfied the claim.
\\\textbf{Non-constructive proof} is when we prove object $X$ exists without 
revealing what $X$ is. 

\section{Proof Errors}
1. Do not assume the claim you aim to prove.
\\Ex: $-2=2\implies (-2)^2=(2)^2 \implies 4=4$ which is true. Therefore $-2 = 2$.
\\2. Don't forget the case where your variables are 0.
\\Ex: accidentally dividing by 0
\\3. Be careful when mixing negatives with inequalities.
\\Ex: you need to flip the inequality if multiply by a negative.

\end{document}