%Setup%
\documentclass[12pt, letterpaper]{article}
\usepackage{cs70}
\title{CS70 Note 3: Induction}
\begin{document}
\maketitle

%Content%
\section{Induction}
Powerful tool to prove a statement holds for all natural numbers.
\\Prove $\forall n\in\N,\sum_{i=0}^ni=\frac{n(n+1)}{2}$
\begin{proof} We do induction on $n$.
\\Base case (n=0): can easily see LHS = RHS = 0 and it works.
\\Induction hypothesis: assume we have proved the statement works for an arbitrary
$n=k\geq0$.
\\Inductive step: prove the statement for $n=k+1$.
$$\sum_{i=0}^{k+1}=\sum_{i=0}^ki+(k+1)=\frac{k(k+1)}{2}+(k+1)=\frac{(k+1)(k+2)}{2}$$
\end{proof}

\section{Strengthening the Induction Hypothesis}
Ex: "Sum of the first $n$ odd numbers is a perfect square"
vs
"sum of the first $n$ odd numbers is $n^2$"
\\The introduction of more constraints/structure was necessary
to create the proof.

\section{Simple vs Strong Induction}
Simple: assume $P(k)$ is true.
\\Strong: assume $P(0),P(1),\dots,P(k)$ are all true.
\end{document}