%Setup%
\documentclass[12pt, letterpaper]{article}
\usepackage{cs70}
\title{CS70 Note 1: Mathematical Foundations}
\begin{document}
\maketitle

%Content%
\section{Propositions}
A statement that's either true or false is called a \textbf{proposition}.
\\Ex: $\sqrt{3}$ is rational, $1+1=5$, CS70 is cool
\\\textbf{Propositional variables} $P,Q,R$ represent arbitrary propositions.
\\\textbf{Connectives} oin propositions together to form more complex ones.
\begin{itemize}
  \item \textbf{And} $P\land Q$
  \item \textbf{Or} $P\lor Q$
  \item \textbf{Not} $\lnot P$
\end{itemize}
Ex: $\sqrt{3}$ is rational $\land$ $1+1=5$
\\\textbf{Propositional formulas} are created by combining propositional variables
with connectives.
\\Ex: $P\land Q\lor \lnot R$

\section{Propositional Logic}
\textbf{Tautology} is a propositional formula that is always true
regardless of the truth values of the variables.
\\\textbf{Contradiction} always false regardless of the truth values
of the variables (tautologically false).
\\\textbf{Truth table} is an algorithm to verify if a propositional
formula is a tautology.
$$\begin{array}{|c|c >{\columncolor{green!15}}c c|}
  P & P & \land & \lnot P\\
  \hline
  T & T & F & F\\
  F & T & T & T\\
\end{array}$$
\textbf{Implication} $P\implies Q$ means "if $P$, then $Q$". 
Only false if $P$ is T and $Q$ is F. Equivalent to
$\lnot P\lor Q$.
\\2 propositional formulas are \textbf{tautologically equivalent} if they 
have the same truth table, written as $P\equiv Q$
\\\textbf{Material equivalence} $P\iff Q$ means
"P if and only if Q". Only true if $P$ and $Q$ are the same.
\\Tautological equivalence allows us to transform 
difficult propositional formulas into equivalent easier ones.
\\\textbf{Contrapositive} 
$\lnot Q\implies\lnot P$ (tautologically equivalent to $P\implies Q$)
\\\textbf{Converse}
$Q\implies P$
\\\textbf{De Morgan's Laws}
$\lnot(P\land Q)\equiv\lnot P\lor\lnot Q$ and 
$\lnot(P\lor Q)\equiv\lnot P\land\lnot Q$

\section{First Order Logic}
\textbf{Predicate} is a function which takes as input
some element from a domain and outputs a proposition.
\\Ex: The predicate statement $x^2+3x=0$ truth depends on the value of x.
\\\textbf{First-order sentence} is a proposition which uses quantifiers.
\\\textbf{Quantifier} quantifies the variable predicate statements use, to turn them into valid propositions.
\begin{itemize}
  \item \textbf{Existence} $\exists xP(x)\sim\bigwedge_{i=1}^{\infty}P(a_i)$
  \item \textbf{Universal} $\forall xP(x)\sim\bigvee_{i=1}^{\infty}P(a_i)$ 
\end{itemize}
\textbf{First-order formulas} are created by combining 
predicate variables and quantifiers. All variables must be bound by a single quantifier to be valid.
\\Predicate variables can represent any predicate, so determining
its truth value is hard. We have to consider all possible predicates (models).
\\\textbf{Model} there are infinitely many per predicate, consists of
\begin{itemize}
  \item \textbf{Domain} a non-empty set of objects and rules that govern them.
  \item An interpretation for all predicates in the first-order formulas.
\end{itemize}
\textbf{Logically true} if every model makes $P$ true.
\\\textbf{Logical implication} $P\implies Q$ if any model that makes $P$ true makes $Q$ true.
\\\textbf{Logical equivalence} $P\iff Q$ if every model that makes one true makes the other true.
\\\textbf{Quantifier rules} allow manipulation.
\begin{itemize}
  \item Universal Instantiation (UI): $\forall xP(x)\implies P(c)$ over any $c$.
  \item Existential Generalization (EG): $P(b)\implies\exists xP(x)$.
  \item Universal Generalization (UG): $P(c)$ for $c$ arbitrary $\implies$ $\forall xP(x)$.
  \item Existential Instantiation (EI): $\exists xP(x)\implies P(b)$ for some $b$.
\end{itemize}
\textbf{Restrict quantifiers} to a certain set of models.
\\Ex: $\exists x(x\in\R\land x^2+3x=0)$.
We short hand this as: $(\exists x\in\R)(x^2+3x=0)$.
\\\textbf{Negate quantifiers} with De Morgan's law.
$$\lnot\forall xP(x)\iff\exists x\lnot P(x) \text{ and } \lnot\exists xP(x)\iff\forall x\lnot P(x)$$
\\\textbf{Numerical quantification} to
get a number between "exists" and "for all".
\\Ex: "At least 2 objects satisfying P"
$$\exists x\exists y(x\neq y\land P(x)\land P(y))$$
\end{document}