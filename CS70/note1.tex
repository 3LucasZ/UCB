%Setup%
\documentclass[12pt, letterpaper]{article}
\usepackage{cs70}
\title{CS70 Note 0: Mathematical Foundations}
\begin{document}
\maketitle

%Content%
\section{Propositions}
A statement that's either true or false is called a \textbf{proposition}.
\\Ex: $\sqrt{3}$ is rational, $1+1=5$, CS70 is cool
\\\textbf{Propositional variables} $P,Q,R$ represent arbitrary propositions.
\\\textbf{Connectives} like and($P\land Q$), or($P\lor Q$), not($\lnot P$) 
join propositions together to form more complex ones.
\\Ex: $\sqrt{3}$ is rational $\land$ $1+1=5$
\\\textbf{Propositional formulas} are created by combining propositional variables
with connectives.
\\Ex: $P\land Q\lor \lnot R$

\section{Propositional Logic}
\textbf{Tautology} is a propositional formula that is always true
regardless of the truth values of the variables.
\\\textbf{Contradiction} always false regardless of the truth values
of the variables (tautologically false).
\\\textbf{Truth table} is an algorithm to verify if a propositional
formula is a tautology.
$$\begin{array}{|c|c >{\columncolor{green!15}}c c|}
  P & P & \land & \lnot P\\
  \hline
  T & T & F & F\\
  F & T & T & T\\
\end{array}$$
\textbf{Implication} $P\implies Q$ means "if $P$, then $Q$". 
Only false if $P$ is T and $Q$ is F. Equivalent to
$\lnot P\lor Q$.
\\2 propositional formulas are \textbf{tautologically equivalent} if they 
have the same truth table, written as $P\equiv Q$


\end{document}