\documentclass[11pt]{article}
\usepackage{header}
\def\title{HW 09}

\begin{document}
\maketitle
\fontsize{12}{15}\selectfont

\begin{center}
    Due: Saturday, 11/2, 4:00 PM \\
    Grace period until Saturday, 11/2, 6:00 PM \\
\end{center}

\section*{Sundry}
Before you start writing your final homework submission, state briefly how you worked on it.  Who else did you work with?  List names and email addresses.  (In case of homework party, you can just describe the group.)

\begin{solution}
  I worked with Lawrence Rhee (lawrencejrhee@berkeley.edu). 
  We discussed our approaches to each problem and asked each other questions.  
\end{solution}

\vspace{15pt}

\Question{Probability Potpourri}

\notelinks{\href{https://www.eecs70.org/assets/pdf/notes/n13.pdf}{Note 13},\href{https://www.eecs70.org/assets/pdf/notes/n14.pdf}{Note 14}}
Provide brief justification for each part.

\begin{Parts}

\Part For two events $A$ and $B$ in any probability space, show that $\Pr[A \setminus B] \geq \Pr[A] - \Pr[B]$.

\Part Suppose $\Pr[D \mid C] = \Pr[D \mid \overline{C}]$, where $\overline{C}$ is the complement of $C$. Prove that $D$ is independent of $C$.

\Part  If $A$ and $B$ are disjoint, does that imply they're independent?

\end{Parts}

\Question{Independent Complements}

\notelinks{\href{https://www.eecs70.org/assets/pdf/notes/n14.pdf}{Note 14}}
Let $\Omega$ be a sample space, and let $A,B \subseteq \Omega$ be two independent events.

\begin{Parts}

\Part Prove or disprove: $\overline{A}$ and $\overline{B}$ must be independent.

\Part Prove or disprove: $A$ and $\overline{B}$ must be independent.

\Part Prove or disprove: $A$ and $\overline{A}$ must be independent.

\Part Prove or disprove: It is possible that $A=B$.

\end{Parts}

\Question{Monty Hall's Revenge}

\notelinks{\href{https://www.eecs70.org/assets/pdf/notes/n13.pdf}{Note 13},\href{https://www.eecs70.org/assets/pdf/notes/n14.pdf}{Note 14}}
Due to a quirk of the television studio's recruitment process, Monty Hall has
ended up drawing all the contestants for his game show from among the ranks of
former CS70 students. Unfortunately for Monty, the former students' amazing
probability skills have made his cars-and-goats gimmick unprofitable for the
studio. Monty decides to up the stakes by asking his contestants to generalize
to three new situations with a variable number of doors, goats, and cars:

\begin{Parts}
    \Part There are $n$ doors for some $n > 2$. One has a car behind it, and the
        remaining $n-1$ have goats. As in the ordinary Monty Hall problem, Monty
        will reveal one door with a goat behind it after you make your first
        selection. Compute the probability of winning if you switch, as well as the probability of winning if you don't switch, and compare the results.

        (\text{Hint:} Think about the size of the sample space for the
        experiment where you \textit{always} switch. How many of those outcomes
        are favorable?)

    \Part Again there are $n > 2$ doors, one with a car and $n-1$ with goats, but
        this time Monty will reveal $n-2$ doors with goats behind them instead
        of just one. How does switching affect the probability of winning in this
        modified scenario?
    \Part Finally, imagine there are $k<n-1$ cars and $n-k$ goats behind the
        $n>2$ doors. After you make your first pick, Monty will reveal $j<n-k$
        doors with goats. What values of $j, k$ maximize the relative
        improvement in your probability of winning if you choose to switch? (i.e. what
        $j, k$ maximizes the ratio between your probability of winning when you switch,
        and your probability of winning when you do not switch?)
\end{Parts}

\Question{Cliques in Random Graphs}

\notelinks{\href{https://www.eecs70.org/assets/pdf/notes/n13.pdf}{Note 13},\href{https://www.eecs70.org/assets/pdf/notes/n14.pdf}{Note 14}}
Consider the graph $G = (V,E)$ on $n$ vertices which is generated by the following random process: for each pair of vertices $u$ and $v$, we flip a fair coin and place an (undirected) edge between $u$ and $v$ if and only if the coin comes up heads.

\begin{Parts}
\Part What is the size of the sample space?

\Part A $k$-clique in a graph is a set $S$ of $k$ vertices which are pairwise adjacent (every pair of vertices is connected by an edge). For example, a $3$-clique is a triangle. Let $E_S$ be the event that a set $S$ forms a clique. What is the probability of $E_S$ for a particular set $S$ of $k$ vertices? 

\Part Suppose that $V_1 = \{v_1, \dots, v_{\ell}\}$ and $V_2 = \{w_1, \dots, w_k\}$ are two arbitrary sets of vertices. What conditions must $V_1$ and $V_2$ satisfy in order for $E_{V_1}$ and $E_{V_2}$ to be independent? Prove your answer.

\Part Prove that $\binom{n}{k} \le n^k$. (You might find this useful in part (e)).

\Part Prove that the probability that the graph contains a $k$-clique, for $k \geq 4{\log_2 n}+1$, is at most $1/n$. \textit{Hint:} Use the union bound.
\end{Parts}

\Question{Symmetric Marbles}

\notelinks{\href{https://www.eecs70.org/assets/pdf/notes/n14.pdf}{Note 14}}
A bag contains 4 red marbles and 4 blue marbles. Rachel and Brooke play a game where they draw four marbles in total, one by one, uniformly at random, without replacement. Rachel wins if there are more red than blue marbles, and Brooke wins if there are more blue than red marbles. If there are an equal number of marbles, the game is tied.
\begin{Parts}
    \Part Let $A_1$ be the event that the first marble is red and let $A_2$ be the event that the second marble is red. Are $A_1$ and $A_2$ independent?
    
    \Part What is the probability that Rachel wins the game?
    
    \Part Given that Rachel wins the game, what is the probability that all of the marbles were red?
\end{Parts}
Now, suppose the bag contains 8 red marbles and 4 blue marbles. Moreover, if there are an equal number of red and blue marbles among the four drawn, Rachel wins if the third marble is red, and Brooke wins if the third marble is blue. All other rules stay the same.
\begin{ResumeParts}
    \Part What is the probability that the third marble is red?
    
    \Part Given that there are $k$ red marbles among the four drawn, where $0 \leq k \leq 4$, what is the probability that the third marble is red? Answer in terms of $k$.
    
    \Part Given that the third marble is red, what is the probability that Rachel wins the game?
    
\end{ResumeParts}

\Question{Socks}
\notelinks{\href{https://www.eecs70.org/assets/pdf/notes/n13.pdf}{Note 13},\href{https://www.eecs70.org/assets/pdf/notes/n14.pdf}{Note 14}}
Suppose you have $n$ different pairs of socks ($n$ left socks and $n$ right socks, for $2n$ individual socks total) in your dresser. 
You take the socks out of the dresser one by one without looking and lay them out in a row on the floor. 
In this question, we'll go through the computation of the probability that no two matching socks are next to each other.

\begin{Parts}
    \Part We can consider the sample space as the set of length $2n$ permutations. What is the size of the sample space $\Omega$, and what is the probability of a particular permutation $\omega \in \Omega$?

    \Part Let $A_i$ be the event that the $i$th pair of matching socks are next to each other. Calculate $\Pr[A_i]$.
    \Part Calculate $\Pr[A_1 \cap  ... \cap A_k]$ for an arbitrary $k \geq 2$. (Hint: try using a counting based approach.)
    \Part Putting these all together, calculate the probability that there is at least one pair of matching socks next to each other. Your answer can (and should) be expressed as a summation. (Hint: use Inclusion/Exclusion.)
    \Part Using your answer from the previous part, what is the probability that no two matching socks are next to each other? (This should follow directly from your answer to the previous part, and also can be left as a summation.)
\end{Parts}

\end{document}
