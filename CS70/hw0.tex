\documentclass[11pt]{article}
\usepackage{header}

\def\title{HW 00}

\begin{document}
\maketitle
\fontsize{12}{15}\selectfont

\begin{center}
    Due: Saturday, 8/31, 4:00 PM \\
    Grace period until Saturday, 8/31, 6:00 PM \\
\end{center}

\section*{Sundry}
Before you start writing your final homework submission, state briefly how you worked on it.  Who else did you work with?  List names and email addresses.  (In case of homework party, you can just describe the group.)

\begin{solution}
  I worked with Lawrence Rhee (lawrencejrhee@berkeley.edu). 
  We discussed our approaches to each problem over facetime. 
  I also consulted the eecs70.org website for information. 
\end{solution}

\vspace{15pt}

\Question{Administrivia}

\begin{Parts}

\Part Make sure you are on the course Ed (for Q\&A) and Gradescope (for submitting homeworks, including this one). Find and familiarize yourself with the course website. What is its homepage's URL?

\Part Read the policies page on the course website.
	\begin{enumerate}[(i)]  
		\item What is the breakdown of how your grade is calculated, for both the homework and the no-homework option?
		\item What is the attendance policy for discussions?
		\item When are homeworks released and when are they due?
		\item How many "drops" do you get for homeworks? How many mini-vitamins will contribute to your grade?
		\item When is the midterm? When is the final?
        \item What percentage score is needed to earn full credit on a homework?
	\end{enumerate}

\end{Parts}

\begin{solution}
\begin{Parts}

\Part https://www.eecs70.org/

\Part 

\begin{enumerate}[(i)]  

\item homework option: 5\% discussion, 5\% vitamin, 15\% HW, 30\% Midterm, 45\% final

non-homework option: 5\% discussion, 5\% vitamin, 36\% Midterm, 54\% Final

\item Attend at least 13 sections for full credit.

May attend any section for credit.

\item Released on Sunday and due the following Saturday at 6:00 pm.

\item 3 HW drops. 13 best vitamins count for a grade.

\item Midterm: 10/15/24 Tuesday 7-9 pm

Final: 12/19/24 Thursday 11:30 am - 2:30 pm

\item 73\% is needed.
\end{enumerate}

\end{Parts}
\end{solution}

\Question{Course Policies}

Go to the course website and read the course policies carefully. Leave a followup on Ed if you have any questions. Are the following situations violations of course policy? Write "Yes" or "No", and a short explanation for each.

\begin{Parts}
  \Part Alice and Bob work on a problem in a study group. They write up a solution together and submit it, noting on their submissions that they wrote up their homework answers together. 
  
  
  \Part Carol goes to a homework party and listens to Dan describe his approach to a problem on the board, taking notes in the process. She writes up her homework submission from her notes, crediting Dan.
  
  
  \Part Erin comes across a proof that is part of a homework problem while studying course material. She reads it and then, after she has understood it, writes her own solution using the same approach. She submits the homework with a citation to the website.
  
  
  \Part Frank is having trouble with his homework and asks Grace for help. Grace lets Frank look at her written solution. Frank copies it onto his notebook and uses the copy to write and submit his homework, crediting Grace.
  
  
  \Part Heidi has completed her homework using \LaTeX. Her friend Irene has been working on a homework problem for hours, and asks Heidi for help. Heidi sends Irene her PDF solution, and Irene uses it to write her own solution with a citation to Heidi.
  
  
  \Part
  Joe found homework solutions before they were officially released, and every time he got stuck, he looked at the solutions for a hint. He then cited the solutions as part of his submission.

\end{Parts}

\begin{solution}
\begin{Parts}

\Part Yes; you can work on problems with other people, but solutions must be written independently.

\Part No; they give the appropriate credit and writes the solution themselves.

\Part No; they give the appropriate credit and doesn't closely copy the solution.

\Part Yes; Frank copies Grace and doesn't write their own solution.

\Part Yes; Irene is pretty much copying from Heidi. Instead, they should discuss to get the solution together.

\Part Yes; using un-released material is a huge violation of honor code and considered cheating.

\end{Parts}
\end{solution}

\Question{Use of Ed}

Ed is incredibly useful for Q\&A in such a large-scale class. We will use Ed for all important announcements. You should check it frequently. We also highly encourage you to use Ed to ask questions and answer questions from your fellow students.

\begin{Parts}
  
    \Part Read the Ed Etiquette section of the course policies and explain what is wrong with the following hypothetical student question: "Can someone explain the proof of Theorem XYZ to me?" (Assume Theorem XYZ is a complicated concept.)

    \Part When are the weekly posts released? Are they required reading?

    \Part If you have a question or concern not directly related to the course content, where should you direct it?

\end{Parts}

\begin{solution}
\begin{Parts}

\Part You need to show what you've done so far to try to understand the concept, instead of asking for an explanation without any context to show you've at least tried.

\Part Weekly posts are released every Monday. It is requierd to read them.

\Part You should email such questions to fa24@eecs70.org.



\end{Parts}
\end{solution}

\Question{Academic Integrity}

Please write or type out the following pledge in print, and sign it.

\begin{quote}
I pledge to uphold the university's honor code: to act with honesty, integrity, and respect for others, including their work. By signing, I ensure that all written homework I submit will be in my own words, that I will acknowledge any collaboration or help received, and that I will neither give nor receive help on any examinations. 
\end{quote}

\begin{solution}
  I pledge to uphold the university's honor code: to act with honesty, integrity, and respect for others, including their work. By signing, I ensure that all written homework I submit will be in my own words, that I will acknowledge any collaboration or help received, and that I will neither give nor receive help on any examinations. 

  Signed: Lucas Jesiah Zheng
\end{solution}

\Question{Propositional Practice}
\notelinks{\href{https://www.eecs70.org/assets/pdf/notes/n1.pdf}{Note 1}}
In parts (a)-(b), convert the English sentences into propositional logic. In parts (c) - (d), convert the propositions into English. For parts (b) and (d), use the notation $a \mid b$ to denote the statement ``$a$ divides $b$'', and use the notation $P(x)$ to denote the statement ``$x$ is a prime number''.
\begin{Parts}

\Part For every real number $k$, there is a unique real solution to $x^3 = k.$

\Part If $p$ is a prime number, then for any two natural numbers $a$ and $b,$ if $p$ doesn't divide $a$ and $p$ divides $ab,$ then $p$ divides $b.$

\Part $(\forall x, y \in \mathbb{R}) \left[(xy = 0) \implies ((x = 0) \lor (y = 0)) \right]$

\Part $\neg((\exists y \in \mathbb{N}) \left[(\forall x \in \mathbb{N}) \left[(x > y) \implies ((y \mid x) \lor P(x))\right] \right])$

\end{Parts}

\begin{solution}
\begin{Parts}

\Part $(\forall k\in\R )(\exists x\in\R)(\forall y\in\R)((x^3=k)\land((y^3=k)\implies(x=y)))$

\Part $(\forall p\in\N)(\forall a\in\N)(\forall b\in\N)(P(p)\implies((\lnot(p\mid a)\land(p\mid ab))\implies(p\mid b)))$

\Part For any 2 real numbers, if their product is 0 then at least one of the numbers is 0.

\Part It is not the case that there exists a natural number y so that 
for every natural number x, if x is greater than y then either y divides x or x is prime. 

\end{Parts}
\end{solution}

\end{document}
