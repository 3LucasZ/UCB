%Setup%
\documentclass[12pt, letterpaper]{article}
\usepackage{cs70}
\title{CS70 Note 0: Mathematical Foundations}
\begin{document}
\maketitle

%Content%
\section{Sets}
Math is built on the foundation of the theory of sets.
In sets, order and count do not matter.
\\Set builder notation:
$\Q=\{\frac{a}{b}:a,b\in\Z,b\neq0\}$
\\Cardinality: the size of a set
$|S|$
\\Subset / superset / proper subset:
$A\subseteq B$ / $A\supseteq B$ / $A\subset B$
\\Natural numbers:
$\N=\{0,1,2,3,\dots\}$
\\Integer numbers:
$\Z=\{\dots,-2,-1,0,1,2,\dots\}$
\\Rational numbers:
$\Q=\{\frac{a}{b}:a,b\in\Z,b\neq0\}$
\\Real numbers:
$\R$
\\Intersection: 
$A\cap B$
\\Union: 
$A\cup B$
\\Relative complement:
$B\setminus A:=\{x\in B:x\notin A\}$
\\Cartesian product:
$A\times B=\{(a,b):a\in A,b\in B\}$
\\Power set:
$2^A=\{S:S\subseteq A\}$
\section{Sums and Products}
$\sum,\prod$
\\Geometric series (infinite), taylor series expansion of the exponential
$$e^x=\sum_{n=0}^\infty\frac{x^n}{n!}$$
\end{document}